\documentclass[12pt,a4paper]{report}

% English
\usepackage[english]{babel} % English language setting
\usepackage[utf8]{inputenc} % Unicode text
\usepackage[T1]{fontenc} % German 'Umlaute'
\usepackage{textcomp} % Euro
\usepackage[hyphens]{url}
\usepackage{amssymb} % Symbols
\usepackage{emptypage} % Empty pages are now actually empty

% Fonts, with all the options
\usepackage{mathpazo}
\usepackage[scaled=.95]{helvet}
\usepackage{courier}
\usepackage{microtype}

% Images and listings
\usepackage{graphicx} % images
\usepackage{subfig} % sub-figures
\usepackage{wrapfig} % wrapping figures
\usepackage{listings} % better source code listings
\usepackage{enumitem}
\graphicspath{ {figures/} }

% Source code
\usepackage{float}
\newfloat{listing}{htbp}{scl}[chapter]
\floatname{listing}{Listing}
\usepackage{packages/coding/golang/lang} % import this package after listings
\usepackage{packages/coding/protobuf/lang}  % include language definition for protobuf
\usepackage{packages/coding/protobuf/style} % include custom style for protobuf declarations.
\lstset{
    basicstyle=\scriptsize\ttfamily,
    columns=[l]flexible,
    mathescape=true,
    showstringspaces=false,
    numbers=left,
    numberstyle=\tiny,
    frame=none,
    keywordstyle=\color{red},
    stringstyle=\color{blue},
    showstringspaces=false,
    tabsize=4,
    xleftmargin=\leftmargin,
    language=Golang
}

% Page layout
\usepackage[paper=a4paper,width=14cm,left=35mm,height=22cm]{geometry}
\usepackage{setspace}
\usepackage[htt]{hyphenat}
\usepackage{sectsty}
\linespread{1.5}
\subsubsectionfont{\large}

% Page markers
\newcommand{\phv}{\fontfamily{phv}\fontseries{m}\fontsize{10}{12}\selectfont}
\usepackage{fancyhdr} % nicer header and footer
\pagestyle{fancy}
\renewcommand{\chaptermark}[1]{\markboth{#1}{}}
\fancyhead[L]{\phv \nouppercase{\leftmark}}
\fancyhead[R]{\phv \thepage}
% rather not use anything for the footer
\fancyfoot[C]{\ } % no page count in the bottom
%\fancyfoot[R]{\textsf{\small Media Management}}

% Share the sources
\usepackage{bibtopic}

% Logo of the university
\usepackage{packages/hsrmlogo}

% Special packages
\usepackage{epigraph}
\setlength{\epigraphrule}{0pt} % no divider
\usepackage{csquotes}

% Some extra styles
\usepackage{soul}
\newcommand*\strikethrough{\st}

% Hyperlink everything
\usepackage{hyperref}
\hypersetup{
    bookmarks=true,
    colorlinks=true,
    linkcolor=black,
    citecolor=black,
    filecolor=black,
    urlcolor=black,
}
\urlstyle{same}

% Wikipedia-style "citation needed" macro
\newcommand{\cn}[1][]{\textsuperscript{\color{red} ~[citation needed]~}}

% Code markup
\newcommand{\code}[1]{\texttt{#1}}


\newcommand{\studyprogramme}{Media Management}
\newcommand{\degreetype}{Bachelor of Arts}
\newcommand{\thesistitle}{
The System and Software Architecture of Applications for the short-term Rental of Electric Scooters
}
\newcommand{\thesissubtitle}{
Analysis of an existing Solution and Evaluation of an exemplarily implemented Back-end Application
}
\newcommand{\thesisauthor}{Jakob Löhnertz}
\newcommand{\thesisdate}{23. Juli 2018}
\newcommand{\thesislocation}{Wiesbaden}
\newcommand{\firstmarker}{Prof.\ Dr.\ Johannes Luderschmidt}
\newcommand{\secondmarker}{Merle Hiort}

\begin{document}

\begin{titlepage}
  \begin{center}
    \hsrmlogo[1]
    \parbox[t]{8cm}{
      Hochschule \textbf{RheinMain}\\
      Fachbereich Design Informatik Medien\\
      Studiengang \studyprogramme}
    \\[1.5cm]
    {\LARGE Abschlussarbeit} \\[0.5cm]
    {\begin{spacing}{1} \large zur Erlangung des akademischen Grades \\[5mm] \end{spacing}}
    {\begin{spacing}{1} \large \degreetype \\[1cm] \end{spacing}}
    \rule{\textwidth}{1pt} \\[0.5cm]
    {\begin{spacing}{1.15} \huge \bfseries \thesistitle \\ \end{spacing}}
    {\begin{spacing}{1.15} \bfseries \thesissubtitle \\[0.60cm] \end{spacing}}
    \rule{\textwidth}{1pt}
    \\[1.5cm]
    \begin{tabular}{ll}
      Vorgelegt von & \thesisauthor \\
      am & \thesisdate \\
      Referent & \firstmarker \\
      Korreferent & \secondmarker
    \end{tabular}
  \end{center}
\end{titlepage}
\cleardoublepage

% Erklärung gemäß den Allgemeinen Bestimmungen für Prüfungsordnungen
\thispagestyle{empty}
\section*{Erklärung gemäß ABPO}
Ich erkläre hiermit, dass ich
\begin{itemize}
\item die vorliegende Abschlussarbeit selbstständig angefertigt,
\item keine anderen als die angegebenen Quellen benutzt,
\item die wörtlich oder dem Inhalt nach aus fremden Arbeiten entnommenen
  Stellen, bildlichen Darstellungen und dergleichen als solche genau
  kenntlich gemacht und
\item keine unerlaubte fremde Hilfe in Anspruch genommen habe.
\end{itemize}

\vspace{6em}
\noindent\begin{tabular}{p{0.37\textwidth}p{0.56\textwidth}}
\thesislocation, \thesisdate  & \rule{0.56\textwidth}{0.5pt}\\
              & \makebox[1cm]{\ } \thesisauthor
\end{tabular}

\vfill

\cleardoublepage



\begin{abstract}

The abstract will be added at this place towards the end of the thesis.

\end{abstract}


\tableofcontents


\chapter{Introduction} \label{chap:intro}


\section{Preamble}

The market for electric scooters is on the rise all over the world,
especially in East Asian countries like China. \cn
In Europe though, this market is still very underrepresented in our modern urban lives.\\
While it becomes sort of a commodity in big East Asian cities, in Europe this vehicle
is still more of a status symbol, partially due to its pricing.\cn \\
The definition of the word \emph{scooter} is fairly vague.
Some describe it as a small vehicle with a handlebar and a platform to stand on,
the kind traditionally children were riding.
This paper looks at the \emph{motorcycle-esque type} of a scooter though, and in
particular the electric versions of them.\\
Many pain points of combustion engines vanish with the switch to their electric
counterparts. Certainly, there are also new challenges that arise.
The maintenance of the mechanical system of the whole vehicle becomes easier.
There are less moving parts which makes for less errors.
On the downside, the exchange of the fuel causes the most problems with electric engines.
Of course they utilize batteries instead of fossil fuels as their impetus.
Although the invention of the battery is older than that of the
combustion engine \cn, the technology is still not perfected and sees new
advancements every year. Alas, it also combats many challenges in terms of
safety, cost, capacity, speed of recharging and longevity.\\
All theses challenges are partly to blame for the inferior sales of electric scooters
outside of East Asia. \cn


\section{Problem}

Follows...


\section{Motivation}

Follows...


\section{Research question}

Follows...


\section{The market for electric scooters}

Follows...


\section{German electric scooter manufacturer \emph{Kumpan}}

This thesis loosely focuses on the business of German electric scooter manufacturer \emph{Kumpan}.
The company was established in 2009. For their product, they concentrated on three main aspects:
an exterior design ready for the European market, build quality and sufficient cruising radius.
They finished building their first electric scooter in 2010 and are thriving for good products ever since.
The price of their electric scooters is still rather high. Owing to that circumstance,
they came up with smaller, cheaper models as well as a completely different approach.\\
Instead of traditional retail sales they started developing the idea of a
short-term rental service for their scooters. And from many angles it makes sense:
As seen in the last section, the market for the short-term rental of electric scooters
especially in Germany is still minuscule compared to e.g. car or bike sharing.
Meanwhile, the costs for buying an electric scooter are still relatively high.
Lastly, many people will not need such a vehicle for the whole day and would benefit
from more flexibility and parking spots are also a scarcity in our inner cities.
Therefore, \emph{Kumpan} plans to enter the market of short-term electric scooter rentals
in the near future in some form or the other.\\
There are still many aspects to be discussed and determined but one of them
is already set and can be seen as a necessity regardless of the final implementation:
A server back-end application as well as its infrastructure.
The handling of the short-term rental only becomes feasible with the automation
of as many processes as possible. This demands such an application ideally paired
with a front-end.


\chapter{Fundamentals} \label{chap:fundamentals}


\section{Components of a modern software product}

The main aspect of success for such a system is indeed the \emph{magic} behind the scenes
that no customer sees. People just see the electric scooter as the product and
want to be able to ride one on demand by just pushing a button.
The ease of use for the customers and their satisfaction are the main facets
to be considered by the provider of such a service.\\
This, as nearly every modern software product geared towards a general audience,
involves two foundational components – a front-end and a back-end.
On the one hand, the front-end provides the user interface of the product, the façade of the whole.
It is the component the customer interacts with but more importantly it will be
what the users identify the product with.
On the other hand, the back-end is the brain of the software product.
It provides endpoints for the front-end to use and consume. The back-end dictates
what the front-end will do. Meanwhile, it does not determine what it will look like.
If the front-end is unusable the back-end will never be able to fix that as the
front-end becomes the mental representation of the whole product for the customers.
If the back-end behaves sluggish or is prone to errors, the front-end will not be able to save much
by the same token. In the end, both form a synergy with each other.\\
This thesis will shed light on the back-end of such an application.
It will focus on the software architecture of it.
This includes the functionalities that are mandatory and the question on how the product
might come together; from ideas and drafts to written source code.


\section{Necessity of a back-end application}

Follows...


\newpage

% In case the source lists are needed:
%\listoffigures
%\listoftables
%\newpage

% Separate the sources with 'bibtopic'
\bibliographystyle{plain}
\begin{btSect}{offline}
\section*{References}
\btPrintCited
\end{btSect}
\begin{btSect}{online}
\section*{Online Sources}
\btPrintCited
\end{btSect}

\end{document}
